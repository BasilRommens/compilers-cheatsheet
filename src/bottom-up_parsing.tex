\section{Bottom-up parsers}
\subsection*{Shift \& Reduce}
It builds a right-most derivation in reverse order. \textbf{Reduce/Reduce:} Top corresponds to handle for 2 different rules. \textbf{Shift/Reduce:} Top corresponds to handle, but parser can shift too.
\subsection*{PDA bottom-up}
Shift: For all terminals a, the PDA has a transition that reads a and pushes a into the stack. Reduce: For every rule $A \rightarrow \alpha_1...\alpha_n$ the PDA has: states $A,\alpha_1,...,\alpha_i$ $\forall 1 \leqslant i < n$ and $(A,\epsilon)$; transition pops letters from $alpha^R$ from the stack without reading from the input. A transition from $(A, \epsilon)$ to push A without reading from input.
\subsection*{Sentential-form prefixes on the stack}
$(q,w,\gamma Z_0)$ with q not of form $(A,...)$ then: $S \Rightarrow^* \gamma^R \cdot w$.$\gamma^R$ happens to be a \textbf{viable prefix}. All the viable prefixes are recognized by the CFSM.
\subsection*{Generalized algorithm action table:}
Same CFSM for LR(0), SLR(K), LALR(k)
\textbf{LR(0)} 1. Push the initial state 0 of the CFSM into the stack S. 2. As long as we can’t accept or get an error: 2.1. If T (top(S)) is Shift, then put the next symbol in variable c. 2.2. Otherwise, if T(top(S)) is \{j\}. 2.2.1. pop $|\alpha|$ times from S, where $A \rightarrow \alpha$ is the j-th rule. 2.2.2. and put A in c. 2.3. Let q' be the next state $\delta (top(S),x)$ of the CFSM after reading c. 2.4. Push  q' into S.\\
\textbf{SLR(k)} 2.1. append ``and the look-ahead starts with a, c=a''. 2.2. Append "$A \rightarrow$ is the j-th rule, and the look=ahead l is in $Follow^k(A)$", lookahead combos from those sets become columns.
\textbf{LALR(k)} same as SLR(k). But uses ItemFollow.\\
\textbf{LALR(1) propagation graph $(V,E,l)$} V contains all state-item-pairs (s,u) st $u \in s$; E contains $(\langle s,u\rangle, \langle s,u\rangle)$ iff 1. t is the closure of the a-successor of s, $u = A \rightarrow \alpha_1 \cdotp a \alpha_2$, and v = $u = A \rightarrow \alpha_1 a \cdotp \alpha_2$. 2. $s = t$ and $u = A \rightarrow \alpha_1 \cdotp B \alpha_2$, $v = B \rightarrow \beta$, $\alpha_2$ can produce $\epsilon$; The labelling function $l: V \rightarrow 2^{\Sigma^{\leqslant 1}}$ is st: $l(\langle(s,B\rightarrow \cdotp \beta))$ = $\cup \{First^1(\alpha_2)$$| \langle s,A \rightarrow \alpha_1 \cdotp B \alpha_2\rangle$ $\in V\}$ and all other vertices are mapped to the empty set.\\
	\textbf{ItemFollow:} For a vertex v, $ItemFollow^1(v)$ is the set $\cup\{ l(u) | u\ has\ a\ path\ to\ v\}$ so this tells us whther we caqn apply reduction rul $A \rightarrow \beta$ from a CFSM state s based on the $ItemFollow^1(v)$ of $v = (s,A\rightarrow \beta \cdotp)$
	\subsection*{LR(k)-CFSM}
	\textbf{LR(k)-items:} $(A\rightarrow \alpha_1 \cdotp \alpha_2,w)$ where $w\in \Sigma^{\leqslant k}$\\
	\textbf{LR(k)-closure:} The closure of $(A\rightarrow \alpha_1 \cdotp \alpha_2,w)$ is the set of all LR(k)-items $(B \rightarrow \cdotp \beta,y)$ where: 1. $B \rightarrow \beta$ grammar rule 2. $y \in First^k(\alpha_2 w)$; It includes I and is minimal set.
	\textbf{CFSM states:} states of the CFSM are now subsets of LR(k)-items. The a-succeswsor of a state I, for $a \in T\cup V$, is the closure of the set $\{(A\rightarrow \alpha_1 a \cdotp \alpha_2,w)$ $|$ $\{(A\rightarrow \alpha_1 \cdotp a \alpha_2,w) \in I\}$
	\textbf{LR(k)-CFSM:} $q_0$ is the closure of $\{(A\rightarrow \alpha_1 a \cdotp \alpha_2,w)$ $|$ $(A\rightarrow \alpha_1 \cdotp a \alpha_2,w) \in I\}$
	\subsection*{Action tables}
	\textbf{LR(0)} Index grammar rules $1 \leqslant j \leqslant n$ and states from the CFSM $0 \leqslant i \leqslant k$ The table T maps each i to a set of actions: 1. $T(i)$ contains (a Reduce) j if state i has the item $A \rightarrow \alpha \cdotp$ with $A \rightarrow \alpha$ the j-th rule. 2. T(i) contains Shift if state i has an item $A \rightarrow \alpha_1 \cdotp \alpha_2$. 3. T(i) contains Accept if state i has an item $S \rightarrow \alpha \cdotp$. 4. $T(\phi)$ contains an error action only.\\
	\textbf{SLR} replace (a Reduce) by \textit{nothing}, replace \textit{Shift} by \textit{(a Shift) a} and prepend an a to $\alpha_2$\\
	\textbf{LR(k):} add and look-ahead l to a set of actions to the 1st line; change to on 1 to $A \rightarrow \alpha \cdotp , l)$; change to on line 2 $(A \rightarrow \alpha_1 \cdotp \alpha_2,y)$ and $l \in First^k(\alpha_2 y)$.
	\subsection*{Conclusions}
$\forall k \geqslant 1$, the LR(k) languages, LR(1) languages, and languages recognizable by DPDA are all the same. (\textbf{Most} languages use LALR(1) grammar)
